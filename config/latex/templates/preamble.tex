\documentclass[11pt, a4paper]{article}
\usepackage[utf8]{inputenc}
\usepackage[T1]{fontenc}
\usepackage{geometry, microtype, xspace}
\usepackage{parskip}
\usepackage[shortlabels]{enumitem}
\usepackage{amsmath, amssymb, amsthm, mathtools, bm, esint}
\usepackage{physics, siunitx}
\usepackage{hyperref}

\geometry{margin=3.0cm}
\sisetup{separate-uncertainty=true, exponent-product=\cdot, range-units=single}
\hypersetup{colorlinks=true, linkcolor=blue, urlcolor=cyan}
\setlength{\parindent}{0pt}
\numberwithin{equation}{section}

\newcommand{\myhref}[2]{\hyperref[#1]{\textcolor{blue}{\underline{#2}}}}
\newcommand{\TODO}[1]{{\textbf{TODO:} {\color{red} #1}}}

% Used to place a label above an equals sign to easily
% reference and explain steps in derivations.
\newcommand{\annotate}[2]{\stackrel{\mathrm{#1}}{#2}}
