% This environment is used for the title page in lecture notes and exercises
% \NewDocumentEnvironment (which requires the xparse package) is used instead of \newenvironment to allow for the year parameter #2 to be used in the environment end code.
% Parameters:
% - title e.g. "Classical Physics"
% - date e.g. "Summer semester 2020-2021"
% The argspec {mm} means the titlepage environment takes two (m)andatory arguments
\usepackage{xparse}
\NewDocumentEnvironment{mytitlepage}{mm}
{% ENVIRONMENT PREAMBLE
\pagestyle{empty}  % remove headers from title page
\begin{center}
\thispagestyle{empty}
\null
\vspace{15ex}
% {\huge \underline{\hphantom{m}{#1}\hphantom{m}}}\\[1.5ex]
{\huge {#1}}\\[2.5ex]

% \setlength{\parindent}{2em}
\begin{minipage}[t]{0.80\textwidth}%
\setlength{\parindent}{1.5em}
    % minipage to hold introductory text
}% ENVIRONMENT CONTENT GOES HERE
{% ENVIRONMENT "POST-AMBLE"
\end{minipage}
% \setlength{\parindent}{0pt}
\vfill

{\textit{For more notes like this, visit \href{https://www.ejmastnak.com/fmf.html}{\texttt{ejmastnak.com/fmf}}}}\\[-1.0ex]
\rule{11cm}{0.4pt}\\[0.5ex]
Elijan J. Mastnak\\
{#2}\\  % use date argument here
Last update: \today\\
Faculty of Mathematics and Physics, University of Ljubljana\\

\end{center}
\newpage
}
% --------------------------------------------- %

% This environment is used for the title page in equation sheets
% \NewDocumentEnvironment (which requires the xparse package) is used instead of \newenvironment to allow for the year parameter #2 to be used in the environment end code.
% Parameters:
% - title e.g. "A Concise Summary of Classical Physics"
% - date e.g. "Summer semester 2020-2021"
% The argspec {mm} means the titlepage environment takes two (m)andatory arguments
\NewDocumentEnvironment{eqnSheetTitlepage}{mm}
{% ENVIRONMENT PREAMBLE
\pagestyle{empty}  % remove headers from title page
\begin{center}
\thispagestyle{empty}
\null
\vspace{5ex}
% {\huge \underline{\hphantom{m}{#1}\hphantom{m}}}\\[1.5ex]
{\huge {#1}}\\[2.5ex]

% \setlength{\parindent}{2em}
\begin{minipage}[t]{0.99\textwidth}%
\setlength{\parindent}{1.5em}
    % minipage to hold introductory text
}% ENVIRONMENT CONTENT GOES HERE
{% ENVIRONMENT "POST-AMBLE"
\end{minipage}
% \setlength{\parindent}{0pt}
\vfill

{\textit{For more notes like this, visit \href{https://www.ejmastnak.com/fmf.html}{\texttt{ejmastnak.com/fmf}}}}\\[-1.0ex]
% \rule{0.9\textwidth}{0.4pt}\\[0.5ex]
% Elijan J. Mastnak\\
% Last update: \today\\
% Faculty of Mathematics and Physics, University of Ljubljana\\

\end{center}
}
% --------------------------------------------- %

% --------------------------------------------- %

% Macro: introduction to lectures
% --------------------------------------------- %
% Params:
% 1: year e.g. "first"/"second"/"third"
% 2: Slovene name e.g. "Kvanta Mehanika"
% 3: English name e.g. "Quantum Mechanics"
% 4: professor name and title e.g. "prof. dr. Anton Ramšak"
% 5: academic year e.g. "2020-21"
\newcommand{\introLecture}[5]{Lecture notes from the {#1}-year undergraduate course \textit{#2} ({#3}) given by {#4} at the Faculty of Mathematics and Physics at the University of Ljubljana in the academic year {#5}}
% --------------------------------------------- %

% Macro: introduction to exercises
% --------------------------------------------- %
% Params:
% 1: year e.g. "first"/"second"/"third"
% 2: Slovene name e.g. "Kvanta Mehanika"
% 3: English name e.g. "Quantum Mechanics"
% 4: professor name and title e.g. "doc. dr. Tomaz Rejec"
% 5: academic year e.g. "2020-21"
\newcommand{\introExercises}[5]{Notes from the Exercises component of the {#1}-year undergraduate course \textit{#2} ({#3}), led by {#4} at the Faculty of Mathematics and Physics at the University of Ljubljana in the academic year {#5}}
% --------------------------------------------- %

% Macro: introduction to oral exam notes
% --------------------------------------------- %
% Params:
% 1: year e.g. "first"/"second"/"third"
% 2: Slovene name e.g. "Kvanta Mehanika"
% 3: English name e.g. "Quantum Mechanics"
% 4: professor name and title e.g. "prof. dr. Anton Ramšak"
% 5: academic year e.g. "2020-21"
\newcommand{\introOral}[5]{Notes covering typical questions from the oral exam required to pass the {#1}-year undergraduate course \textit{#2} ({#3}), given by {#4} at the Faculty of Mathematics and Physics at the University of Ljubljana in the academic year {#5}}
% --------------------------------------------- %

% Macro: introduction to equation sheet
% --------------------------------------------- %
% Params:
% 1: year e.g. "first"/"second"/"third"
% 2: Slovene name e.g. "Kvanta Mehanika"
% 3: English name e.g. "Quantum Mechanics"
% 4: academic year e.g. "2020-21"
\newcommand{\introEqSheet}[5]{Condensed notes of the material covered in the {#1}-year undergraduate course \textit{#2} ({#3}) at the Faculty of Mathematics and Physics at the University of Ljubljana in the academic year {#4}}
% --------------------------------------------- %

% --------------------------------------------- %
\newcommand{\creditMaterial}[1]{Credit for the material covered in these notes is due to #1}
\newcommand{\creditElijanUpper}{The voice, typesetting, and translation to English in this document are my own}
\newcommand{\creditElijanLower}{the voice, typesetting, and translation to English in this document are my own}
\newcommand{\mistakesLine}{\vspace{0.5ex}\textit{Disclaimer:} This document will inevitably contain some mistakes---both simple typos and legitimate errors.
Keep in mind that these are the notes of an undergraduate student in the process of learning the material himself, so take what you read with a grain of salt.
If you find mistakes and feel like telling me, I will be grateful and happy to hear from you, even for the most trivial of errors.
You can reach me by email, in English, Slovene, or Spanish, at \href{mailto:ejmastnak@gmail.com}{\texttt{ejmastnak@gmail.com}}.}
% \href{https://github.com/ejmastnak/fmf}{\underline{Github}}
